\documentclass{llncs}

\usepackage{standalone}
\input{preamble.tex}
\input{macros.tex}

\title{%
	Automatically Translating Proof Systems
	for SMT Solvers to the λΠ-calculus}
\author{Ciarán Dunne, Guillaume Burel}
\institute{INRIA, ENS Paris-Saclay}

\begin{document}
\maketitle
% we're not just translating proofs, but also the proof systems embedded in Eunoia.

\begin{abstract}
	\noindent
	Eunoia is a logical framework used formalizing the
	proof production facilities of SMT solvers.
	%
	We present an encoding of Eunoia signatures and theories
  into the \emph{$λΠ$-calculus modulo rewriting} as
  implemented by the $\lambdapi$ proof assistant.
	%
	Our encoding is demonstrated by the development
	of a tool \texttt{eo2lp}, which we used for
	(a) translating the portion of \texttt{cvc5}'s
	\textit{co-operating proof calculus} (CPC)
	corresponding to the QF-UF fragment of SMT-LIB; and
	%
	(b) translating proofs produced by running \texttt{cvc5}
	on a set of QF-UF problems from the SMT-LIB
	benchmark library.
\end{abstract}

\section{Background}
\input{intro.tex}

\section{Eunoia}\label{sec:eunoia}
\input{eoas.tex}

\section{λΠ-calculus modulo rewriting}\label{sec:lambdapi}
\input{lp.tex}

\section{Translation and Results}\label{sec:results}
\documentclass[class=llncs, crop=false]{standalone}

\input{preamble.tex}
\input{macros.tex}

\begin{document}
% ------------------------
Recall the definition of Eunoia terms and commands
given by \autoref{sec:eunoia}, and similarly those
of $\lambdapi$ from \autoref{sec:lambdapi}.
%
We define a \emph{translation} operator below,
which may act on the terms, types, and commands
of Eunoia.
%
First, we define an injection from Eunoia symbols
to those of $\lambdapi$.
%
\begin{definition}
For any $\eunoia$ symbol $s ∈ 𝒮_\eo$, define $\bar s$ thus:
$$
  \bar s =
\begin{cases}
  \esc s
  & \text{if $s$ contains any of
    $\{ \mtt{\$}, \mtt{@}, \ldots \}$,}
  \\
  {\ s}
  & \text{otherwise.}
\end{cases}
$$
\end{definition}
%
\begin{definition}
%
Let $\tm{⋅}$ be the least operator such that
$\tm{s} = \bar s$ and:
%
$$
\begin{array}[t]{r@{\ =\ }l}
  \tm{\eapp{s}{\vec t}}
  &
  \begin{cases}
    \tm{t_1} ⤳ \ldots ⤳ \ty{t_n}
    &
    \text{if $s = \prn{\mtt{->}}$,}
  \\
    \appldots{\bar s}{\tm{t_1}}{\tm{t_n}}
    &
    \text{otherwise.}
\end{cases}
\end{array}
$$
%
Furthermore, let $\ty{⋅}$ be the least operator
such that:
$$
\begin{array}[t]{r@{\ =\ }l}
  \ty{s} &
  \begin{cases}
    \Set & \text{if $s = \mtt{Type}$,}
  \\
    \El{\tm{s}} & \text{otherwise.}
  \end{cases}
\\[5mm]
  \ty{\eapp{s}{\vec t}}
  &
  \begin{cases}
    \ty{t_1} → \ldots → \ty{t_n}
    &
    \text{if $s = \prn{\mtt{->}}$,}
  \\
    \El{\tm{\eapp{s}{\vec t}}}
    &
    \text{otherwise.}
\end{cases}
\end{array}
$$
%
\end{definition}
%
Now, recall the abstract interface for $\eunoia$ commands
defined in \autoref{sec:eo-judge},
and also those of $\lambdapi$.
%
\begin{definition}
Let $\cmd{⋅}$ be the least operator such that
for any standard command $δ$,
$\cmd δ$ is a set of $\lambdapi$ commands satisfying
the following:
$$
\begin{array}[t]{c}
δ ⊢ s(\vec ρ) : t
{\quad{⟹}\quad}
\begin{cases}
\prn
{\mtt{symbol}\ {\bar s}\ {\ctx{\vec ρ}} : {\ty{t}} ≔ \tm{t'}}
∈ \cmd{δ} & \text{if $δ ⊢ s(\vec ρ) ≔ t'$,}
\\
\prn{\mtt{symbol}\ {\bar s}\ {\ctx{\vec ρ}} : {\ty{t}}}
∈ \cmd{δ}
& \text{otherwise.}
\end{cases}
\\[5mm]
\delta = \epar{\mtt{include}\ μ}
{\quad{⟹}\quad}
\cmd{δ} = \{\prn{\mtt{require}\ \mtt{open}\ {μ}}\}
\end{array}
$$
%
The translation on proof commands $\cmd{\pi}$ is defined
similarly, where assumptions and steps are mapped to symbol declarations
in $\lambdapi$ that reflect their names, conclusions, and (if applicable)
rules, premises, and arguments, using the judgements $\pi ⊢ s : \eapp{\mtt{Proof}}{φ}$
and $\pi ⊢ s ≔ \eapp{s'}{\plur t m\ \plur p n}$.
\end{definition}

We have implemented this translation as an OCaml program called
\texttt{eo2lp}, which uses the Menhir parser generator for parsing
Eunoia signatures and proof scripts. The tool reads Eunoia files,
elaborates them using the operator from \autoref{sec:eo-elab},
applies the translation operators defined above, and outputs
corresponding $\lambdapi$ code.


\subsubsection{Testing and Benchmarks.}
%
Because Eunoia is a complex and evolving system, we created a fork
of the CPC signature called CPC-MINI, which corresponds to the
fragment of CPC needed for formalizing the constants and inference
rules used in \cvc{5} proofs whose input problems are from the QF-UF
fragment of SMT-LIB. We translated all of CPC-MINI into $\lambdapi$
code, mirroring the directory tree of CPC-MINI.

For proof scripts, we used a small benchmark library (called `rodin')
of 30 unsatisfiable problems from the SMT-LIB benchmark suite,
restricted to the QF-UF fragment. We ran \cvc{5} on these problems
with the option \texttt{--proof-format=cpc} to dump their proofs in
Eunoia format. We then ran our \texttt{eo2lp} tool on these proof
scripts and obtained $\lambdapi$ files that typecheck successfully.

% \section{Conclusion and Future Work}\label{sec:conclu}

% \begin{definition}
% Let $\cmd{⋅}$ be the least operator such that:
% %
% $$
% \begin{array}[t]{l@{\ =\ }l}
% \cmd{\epar{\mtt{declare-const}\ s\ t\ \maybe{α}}}
% & \mtt{symbol}\ {\bar s} : {\ty{t}};
% \\
% %-----------
% \cmd{\epar{\mtt{declare-parameterized-const}\ s\ \epar{\vec{ρ}}\ t\ \maybe{α}}}
% & \mtt{symbol}\ {\bar s}\ {\ctx{\vec ρ}}: {\ty{t}};\
% \\
% %-----------
% \cmd{\epar{\mtt{define}\ {s}\ \epar{\vec ρ} \ t\ \maybe{\mtt{:type}\ t'}}}
% & \mtt{symbol}\ {\bar s}\ {\ctx{\vec ρ}} : {\ty{t'}}
%   ≔ \tm{t};
% \\
% %-----------
% \cmd{
% \epar{\mtt{program}\ {s} \ \epar{\vec ρ}\
% 			    \mtt{:signature}\,\epar{\vec{t}}\,t'
% 				\
% 			    \epar{\vec{r}}
% 			}
% }
% & \mtt{symbol}\ {\bar s}\ {\ctx{\vec ρ}} : {\ty{\earr{\vec t\ t'}}};

% \end{array}
% $$
% $$
% \begin{array}[t]{c@{\ }l}
% 			    {⦂} &                                \\
% 			    {∣} &  \\
% 			    {∣} &    \\
% 			    {∣} &                                                                           \\
% 			    {∣} & {\color{MaterialGrey600}{\texttt{(}}}
% 			    \mtt{declare-rule}\ s\ \epar{\vec ρ}\
% 			    \\ & \quad \maybe{\mtt{:premises}\ \epar{\vec{t}_{\text{prem}}}}\
% 			    \\ & \quad \maybe{\mtt{:args}\ \epar{\vec{t}_{\text{args}}}}\
% 			    \\ & \quad \maybe{\mtt{:requires}\ \epar{\vec{r}}}\
% 			    \\ & \quad \mtt{:conclusion}\ {t_{\text{conc}}}
% 			    {\color{MaterialGrey600}{\texttt{)}}}                                           \\
% 			    {∣} & \epar{\mtt{include}\ μ}
% 		    \end{array}
% $$

\end{document}


\section{Conclusion and Future Work}\label{sec:conclu}

We have presented a translation from Eunoia specifications and proofs
to the $\lambda\Pi$-calculus, as implemented in the $\lambdapi$ proof
assistant. Our tool \texttt{eo2lp} automates this translation and has
been demonstrated on a subset of the CPC signature and proofs from
QF-UF benchmarks.

It should be stressed that the presentation of Eunoia in this paper
only represents the core features, and there is more work to do in
order to (a) translate all of the CPC signature and therefore be able
to translate arbitrary \cvc{5} proofs, and (b) robustly cover the
entirety of Eunoia to be able to translate arbitrary Eunoia signatures
and proof scripts.

We should also aim to support larger proofs, as those from the rodin
benchmark have fewer than 70 proof steps. Some \cvc{5} proofs can be
massive, so computational and efficiency aspects of the translation
must be investigated.

For future work, we should also investigate using our translated
signatures and proof scripts in the realm of proof interoperability.
For example, to gain even more confidence in the results of \cvc{5},
one could attempt to prove the consistency of CPC, either within
$\lambdapi$ or by exporting to some other proof assistant like
Isabelle/HOL or Rocq.

\printbibliography

\end{document}


% \hrule
% \vspace{2mm}

% 	\hrule
% $$\begin{array}{r@{\ }l@{\quad}l}
% 		δ &
% 		\begin{array}[t]{c@{\ }l}
% 			{⦂} & \prn{\decl{x}{\prn{\vec ρ}}{\prn{{:}\ t}}{\prn{{∷}\ α}}}
% 			∣ \prn{\defn{x}{\paren{\vec ρ}}{\paren{{≔}\ t}}{\paren{{:}\ t'}}}    \\
% 			{∣} & \paren{\prog{x}
% 			{\paren{\vec ρ}}
% 			{\paren{{:}\ {\paren{t_1 \ldots t_n} → t'}}}
% 			{\paren{r_1 \ldots r_m}}
% 			}                                                                      \\
% 			{∣} & \paren{\irule{x}
% 				{\paren{\vec ρ}}
% 				{\paren{φ_1 \ldots φ_n}}
% 				{\paren{t_1 \ldots t_m}}
% 				{\paren{ψ}}
% 			}
% 		\end{array}
% 		  & \mcomment{(declarations)}
% 	\end{array}
% $$
